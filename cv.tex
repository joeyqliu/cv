% !TEX TS-program = lualatex

\documentclass[a4paper,10pt]{article}

\usepackage[usenames,dvipsnames]{xcolor}
\usepackage{libertine}
\usepackage{fontawesome5}
\usepackage{array}
\usepackage{longtable}
\setlength{\LTpost}{0pt}
\usepackage{parskip}
\usepackage[cm]{fullpage}

% headers and footers
\usepackage{fancyhdr}
\usepackage{lastpage}
% set release number (default if not provided)
\ifdefined\releasenumber
  \def\getreleaseversion{\releasenumber}
\else
  \def\getreleaseversion{undefined}
\fi
% set release url (default if not provided)
\ifdefined\releaseurl
  \def\getreleaseurl{\releaseurl}
\else
  \def\getreleaseurl{https://github.com/joeyqliu/cv/releases/}
\fi
\pagestyle{fancy}
\fancyfoot[L]{Joey Liu}
\fancyfoot[C]{Page~\thepage~of~\pageref*{LastPage}}
\fancyfoot[R]{Release \href{\getreleaseurl}{\getreleaseversion}~--~\today}
\renewcommand{\footrulewidth}{0.4pt}% default is 0pt
\fancyhead{}
\renewcommand{\headrulewidth}{0pt}

% StackOverflow-like tags
% https://tex.stackexchange.com/a/311949/142692
% https://tex.stackexchange.com/questions/387499/how-to-create-a-border-that-looks-like-a-tag
\usepackage{tikz}
\ifdefined\darkmode
  \definecolor{tagbg-blue}{RGB}{120,180,240}
  \definecolor{tagbg-green}{RGB}{120,240,180}
  \definecolor{tagtxt-blue}{RGB}{0,0,0.1}
  \definecolor{tagtxt-green}{RGB}{0,0.1,0}
\else
  \definecolor{tagbg-blue}{RGB}{225,236,244}
  \definecolor{tagbg-green}{RGB}{225,244,235}
  \definecolor{tagtxt-blue}{RGB}{88,115,159}
  \definecolor{tagtxt-green}{RGB}{87,158,122}
\fi
\newcommand{\sotag}[1]{\tikz[baseline]{\node[anchor=base, rounded corners=0.5ex, text height=1.5ex, text depth=.25ex, fill=tagbg-blue, draw=tagbg-blue, text=tagtxt-blue] {#1};}}
\newcommand{\sotagarch}[1]{\tikz[baseline]{\footnotesize\node[anchor=base, inner sep=2pt, rounded corners=0.5ex, text height=1.5ex, text depth=.25ex, fill=tagbg-green, draw=tagbg-green, text=tagtxt-green] {#1};}}
\newcommand{\sotagtech}[1]{\tikz[baseline]{\footnotesize\node[anchor=base, inner sep=2pt, rounded corners=0.5ex, text height=1.5ex, text depth=.25ex, fill=tagbg-blue, draw=tagbg-blue, text=tagtxt-blue] {#1};}}

% Helpers for adding job entries
\usepackage[english]{babel}
\usepackage[autostyle=try]{csquotes}
\usepackage[en-GB,calc,useregional,useregional=numeric,datesep=/]{datetime2}
\DTMnewdatestyle{shortmonth}{%
 \renewcommand{\DTMdisplaydate}[4]{%
  \DTMshortmonthname{##2} \number##1 }%
 \renewcommand{\DTMDisplaydate}{\DTMdisplaydate}%
}
\newcommand{\displayshortmonth}[1]{%
{% group actions
  \DTMsetdatestyle{shortmonth}%
  \DTMsavedate{mydate}{#1}\DTMUsedate{mydate}%
}%
}%
\newcount\datediffdays%
\newcounter{diffdays}
\newcommand{\setdatediffdays}[2]{%
  \DTMsavedate{startdate}{#1}%
  \DTMsavedate{enddate}{#2}%
  \DTMsaveddatediff{enddate}{startdate}{\datediffdays}%
  \setcounter{diffdays}{\number\datediffdays}%
  \ifnum\value{diffdays}<0
    \setcounter{diffdays}{-\value{diffdays}}%
  \fi
}
\newcommand{\displaydaysdiff}[2]{%
  \setdatediffdays{#1}{#2}%
  \thediffdays\space day\ifnum\value{diffdays}>1s
}
\usepackage{calc}
\newcounter{diffyears}
\newcounter{diffmonths}
\newcommand{\displaymonthsdiff}[2]{%
  \setdatediffdays{#1}{#2}%
  \setcounter{diffyears}{\value{diffdays}/\real{365.25}}%
  \setcounter{diffdays}{\value{diffdays}-\value{diffyears}*\real{365.25}}%
  \setcounter{diffmonths}{\value{diffdays}/\real{30.43}}%
  \setcounter{diffdays}{\value{diffdays}-\value{diffmonths}*\real{30.43}}%
  %
  \ifnum\value{diffyears}=0
  \else
    \thediffyears\space year\ifnum\value{diffyears}>1s\fi
    \ifnum\value{diffmonths}>0, \fi
  \fi
  \ifnum\value{diffmonths}=0
  \else
    \thediffmonths\space month\ifnum\value{diffmonths}>1s\fi
  \fi
}
\newcommand{\displayyearsdiff}[2]{%
  \setdatediffdays{#1}{#2}%
  \setcounter{diffyears}{\value{diffdays}/\real{365.25}}%
  \ifnum\value{diffyears}=0
  \else
    \thediffyears\space year\ifnum\value{diffyears}>1s\fi
    \ifnum\value{diffmonths}>0, \fi
  \fi
}
\usepackage{makecell}
\newcommand{\joblog}[7]{
  \makecell[tr]{%
    \textsc{\displayshortmonth{#4}}%
    \setdatediffdays{#5}{\DTMfetchyear{now}-\DTMfetchmonth{now}-\DTMfetchday{now}}%
    \,\faLongArrowAltRight{}%
    \ifnum\value{diffdays}=0%
      \textbf{\textit{Present}}%
    \else%
      \textsc{\displayshortmonth{#5}}%
    \fi\smallskip\\%
    \textit{(\displaymonthsdiff{#4}{#5})}%
  }%
  &\makecell[t{{p{14cm}}}]{%
    \large\sffamily \textbf{#1} at \textbf{#2}, \small{#3}\smallskip\\%
    \ifx&#7&%
    \else%
      #7\smallskip\\%
    \fi%
    #6%
  }\\\multicolumn{2}{c}{}\\
}

\DTMsavenow{now}
\newcommand{\joblogcurrent}[5]{\joblog{#1}{#2}{#3}{#4}{\DTMfetchyear{now}-\DTMfetchmonth{now}-\DTMfetchday{now}}{#5}}
\newcommand{\displayage}[1]{\displayyearsdiff{#1}{\DTMfetchyear{now}-\DTMfetchmonth{now}-\DTMfetchday{now}} old}

% tweak the url colors
\usepackage{hyperref}
\ifdefined\darkmode
  \definecolor{linkcolor}{rgb}{0.6,0.8,1}
\else
  \definecolor{linkcolor}{rgb}{0,0.2,0.6}
\fi
\hypersetup{colorlinks,breaklinks,urlcolor=linkcolor,linkcolor=linkcolor}

% nicer-looking section titles
\usepackage{titlesec}
\titleformat{\section}{\Large\scshape\raggedright}{}{0em}{}[\titlerule]
\titlespacing{\section}{%
  0pt}{%left margin
  1em}{%space before (vertical)
  3pt}%space after (horizontal)
  
\renewcommand{\arraystretch}{1.3}
\ifdefined\darkmode
  \definecolor{darkergray}{gray}{0.15}
  \pagecolor{darkergray}
  \color{white}
\fi
\begin{document}

\par{
  \centering{
    \Huge \textsc{Joey Liu}\par
    \large \textsc{Software Engineer}\par
  }
  \bigskip\par
}

\hrule
\vspace{0.5em}

\begin{center}
\newcolumntype{x}[1]{>{\centering\arraybackslash}m{#1}}
\begin{tabular}{ cx{14em} cx{14em} cx{14em} }
  \textsc{Contact} 
  & \textsc{Socials}\\
  \faEnvelope[regular]{} \href{mailto:joeyqliu@gmail.com}{joeyqliu@gmail.com}
  & \faLinkedin{} \href{https://www.linkedin.com/in/joeyqliu/}{linkedin}\\
  \faMobile*{} \href{tel:5104088928}{(510) 408-8928} 
  & \faGithub{} \href{https://github.com/joeyqliu}{github}
\end{tabular}
\end{center}

\section{Summary}
\begin{tabular}{p{0.97\textwidth}}
  I am a passionate software engineer with a deep understanding of platform engineering. My expertise lies in managing large-scale deployments of infrastructure-as-code platforms, 
  and I have hands-on experience with various cloud platforms, including AWS and GCP. I take pleasure in empowering fellow engineers to deploy infrastructure and excel in bridging 
  the gap between product teams and operations. \medskip

  Interests: \sotag{infrastructure-as-code} \sotag{aws} \sotag{gcp} \sotag{kubernetes} \sotag{python} \sotag{policy-as-code} \sotag{terragrunt} \sotag{golang}
\end{tabular}

\section{Experience}
\begin{longtable}{p{3.2cm}|p{14.1cm}}
  \joblogcurrent{Software Engineer II, Cloud Engineering}{Handshake}{San Francisco, CA}{2023-05-01}{%
    • Researched, conducted proof-of-concepts, and orchestrated the migration of over 200 Terraform workspaces from Terraform Cloud to Spacelift in less than two months, resulting in an annual savings of over \$100,000 in SaaS costs. \\%
    • 
    • Identified inefficiencies in our Terraform codebase and collaborated with all engineering teams to standardize and unify our repository, saving engineers more than 2 hours per week in deploying infrastructure.\\%
    • Implemented workload identity federation across all GCP projects, enabling dynamic credential generation and reducing over 250 outdated service account keys.\\%
  }{}

  \joblog{Software Engineer II, Platform Engineering}{Rivian}{Palo Alto, CA}{2021-03-01}{2023-03-01}{%
  • Maintainer and quality controller of our collaborative infrastructure as code technology stack which supported 500+ engineers.\\%
  • Migrated script based helm deployments to a Terraform module that enabled manageable, deployable, version-controlled deployments adopted across 20+ Kubernetes clusters.\\%
  • Reimplemented a system-critical DNS component within Kubernetes and led a seamless migration of it through multiple environments across 8+ EKS clusters.\\%
  • Automated tech writing publication process resulting in automated publication of 200+ documents saving 2 hours a week of manual submissions.\\%
  • Built tools to measure AWS infrastructure drift within 8+ AWS accounts resulting in in depth understanding of unmanaged and orphaned infrastructure. \\%
  • Contributed bug fixes to industry wide open source projects such as Open Policy Agent and EKS-Blueprints for Kubernetes.\\%
  • Maintained and orchestrated 10+ clusters across multiple regions and environments.\\%
  • Automated gitlab pipeline scripts for routine repository updates from an outside provider saving 3+ hours of manual updates.\\%
  • Developed bash script to migrate Kubernetes horizontal node scaling from cluster-autoscaler to Karpenter for EKS saving 2+ hours of manual operation per EKS cluster.\\%\\%
  • Designated point of contact platform engineering liaison for 3+ 15 person engineering teams to guide and advise throughout their design and build phases.\\%
  • Designed, maintained, and hardened our repository of over 50 in house Terraform modules utilized by 500+ engineers for their cloud infrastructure.\\%
  • Automated and standardized the documentation for in house Terraform modules.\\%
  • Participated in on call rotation for all critical platform related services.\\%
  }{}
  
  \joblog{Software Engineer}{IoTium}{Santa Clara, CA}{2019-10-01}{2021-01-15}{
  • DevOps / Infrastructure / Automation software engineer working on the OT Access solution at Iotium. \\%
  • Built, wrote, and executed automation framework that deploys our microservices to development, staging, and production environments utilizing Python and Ansible. \\%
  • Mitigated potential downtimes to approximately 10 minutes through a database rollback feature. \\%
  • Developed AWS infrastructure automation through Terraform and Python. \\%
  • Designed and automated custom RBAC in AWS for internal teams to manage, monitor, and troubleshoot AWS infrastructure. \\%
  • Implemented CLI features that allowed for bulk onboarding of devices using YAML format. \\%
  • Shortened release processes to all production environments to 30 minutes using scripts written in Bash and Python. \\%
  • Refined and developed internal tools and scripts to reduce operations team workload by over 20\%. \\%
  • Utilized Jenkins to continuously build and release newest images of our microservices. \\%
  • Point of contact person between support, solutions, and US to India engineering teams, furthering communication and understanding of requirements for all sides. \\%
  • Experienced in working and debugging services within a Linux environment. \\%
  • Participated in on call rotation to mitigate any customer issues. \\%
  }{}
\end{longtable}

\section{Education}
\begin{longtable}{p{3.2cm}|p{14.1cm}}
  \joblog{B.S. Computer Science}{University of California, Santa Cruz}{Santa Cruz, CA}{2015-09-20}{2019-06-15}
  • Led a group of college students and post grads from the US to Taiwan for a 2 week cultural exchange program.\\%
  • Mentored and guided students through their first time abroad.\\%
  • Organized and led cultural exchange activities between US college students and Taiwanese students.\\%  
  }{}
  \joblog{Cultural Exchange Mentor}{Camp Blue}{Hsinchu, Taiwan}{2018-08-01}{2018-09-01}{%}
  • Volunteered for an English cultural exchange camp hosted by National Tsing Hua University and National Chiao Tung University. \\% 
  • Led a group of 6 students in discussions, talks, and social gatherings. \\% 
  • Hosted a talk on the effects and influence of social media and technology in our lives.\\%  
  }{}
  \joblog{Mentor}{Life University}{Sihanoukville, Cambodia}{2014-06-01}{2014-07-02}{%
  • Taught English to elementary and high school students at Life University in Cambodia.\\%
  • Visited rural villages with songs, skits, and dance to engage with the kids and villagers.\\%
  }{}
\end{longtable}

\section{Skills}
\begin{tabular}{r|l}
  \textsc{Infrastructure}& Terraform -- Spacelift -- Terragrunt -- Ansible -- Terraform Cloud\\
  \textsc{Cloud}& AWS -- GCP\\
  \textsc{Containers \& Orchestration:}& Kubernetes -- EKS -- GKE -- Docker\\
  \textsc{Product management:}& Linear -- Scrum -- Kanban -- Jira -- Confluence\\
  \textsc{IDE:}& VS Code \\
  \textsc{Version Control:}& Git -- GitHub -- GitLab\\
  \textsc{Scripting:}& Bash -- Zsh -- Python\\
  \textsc{Programming:}& Python -- Java -- Golang -- LaTeX -- OPA\\
\end{tabular}

\section{Languages}
\begin{center}
  \newcolumntype{x}[1]{>{\centering\arraybackslash}m{#1}}
  \begin{tabular}{ cx{14em} cx{14em} cx{14em} }
    \textsc{English} 
    & \textsc{Chinese}\\
    Native
    & Working Proficiency\\
  \end{tabular}
\end{center}

\end{document}
